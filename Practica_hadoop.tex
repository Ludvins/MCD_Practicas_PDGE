% Created 2020-09-29 mar 13:20
% Intended LaTeX compiler: pdflatex
%%% Local Variables:
%%% LaTeX-command: "pdflatex --shell-escape"
%%% End:
\documentclass[11pt]{article}
\usepackage[utf8]{inputenc}
\usepackage[T1]{fontenc}
\usepackage{graphicx}
\usepackage{grffile}
\usepackage{longtable}
\usepackage{wrapfig}
\usepackage{rotating}
\usepackage[normalem]{ulem}
\usepackage{amsmath}
\usepackage{textcomp}
\usepackage{amssymb}
\usepackage{capt-of}
\usepackage{hyperref}
\hypersetup{colorlinks=true, linkcolor=black}
\setlength{\parindent}{0in}
\usepackage[margin=1.2in]{geometry}
\usepackage[spanish]{babel}
\usepackage{mathtools}
\usepackage{palatino}
\usepackage{fancyhdr}
\usepackage{sectsty}
\usepackage{engord}
\usepackage{cite}
\usepackage{graphicx}
\usepackage{setspace}
\usepackage[compact]{titlesec}
\usepackage[center]{caption}
\usepackage{placeins}
\usepackage{color}
\usepackage{amsmath}
\usepackage{minted}
\usepackage{pdfpages}
\titlespacing*{\subsection}{0pt}{5.5ex}{3.3ex}
\titlespacing*{\section}{0pt}{5.5ex}{1ex}
\author{Antonio Coín Castro\\ Luis Antonio Ortega Andrés}
\date{\today}
\title{Práctica Hadoop\\\medskip
\large Procesamiento de Datos a Gran Escala}
\begin{document}

\maketitle

\section*{Solución WordCount}

Para hacer que el fichero \texttt{.jar} generado por el script tenga otro nombre debemos considerar un argumento mas (\texttt{\$2}) y realizar el siguiente cambio en la última línea de ejecución del script:

\begin{minted}{sh}
jar -cvf $2.jar -C ${file} .
\end{minted}


Para conseguir que el programa WordCount que hemos tomado no tenga en cuenta los signos de puntuación ni las mayúsculas y minúsculas, debemos realizar dos cambios distintos:
\begin{itemize}
  \item Para evitar los signos de puntuación, los añadimos al conjunto de delimitadores de palabras. Para ello configuramos el parámetro de delimitadores de la clase \texttt{StringTokenizer} de \texttt{Java}.
  \item Para no distinguir mayúsculas y minúsculas, pasamos cada uno de los caracteres de cada palabra a minúscula utilizando la función \texttt{toLowerCase()}.
\end{itemize}
  \begin{minted}{java}
        StringTokenizer itr = new StringTokenizer(value.toString()
                   .toLowerCase(), " \n\t\r\f.,;:-¡¿?!()\"\'");
\end{minted}

En la solución obtenida al ejecutar directamente los ejemplos hadoop map-reduce (archivo \texttt{Resultados\_ejemplo}) podemos ver como no se utilizan separadores correctos, diferenciando por ejemplo las palabras \texttt{que} y \texttt{(que}. Ante esto, los resultados obtenidos mediante la versión modificada resultan ser más precisos.

\subsection*{Cuestiones planteadas}

\textbf{Pregunta 1. }\textit{¿Dónde se crea hdfs? ¿Cómo se puede elegir su localización?}\\

La localización del sistema de archivos HDFS la determina la variable \texttt{dfs.datanode.data.dir}, cuyo valor por defecto según la documentación de \href{https://hadoop.apache.org/docs/r2.4.1/hadoop-project-dist/hadoop-hdfs/hdfs-default.xml}{hadoop} es \texttt{file://\${hadoop.tmp.dir}/dfs/data}

Este valor se puede cambiar en el archivo \texttt{hdfs-site.xml}:

\begin{minted}{html}
<property>
  <name>dfs.datanode.data.dir</name>
  <value>file:/hadoop/hdfs/datanode</value>
</property>
\end{minted}

\textbf{Pregunta 2. }\emph{Si estás utilizando hdfs, ¿cómo puedes volver a ejecutar WordCount como si fuese single.node?}\\

Para volver a configurar hadoop para funcionar como single.node debemos revertir los cambios que hicimos en los archivos de configuración para habilitar HDFS. Es decir, eliminar las secciones \texttt{fs.defaultFS} del archivo \texttt{core-site.xml} y \texttt{dfs.replication} de \texttt{hdfs-site.xml}.\\


\textbf{Pregunta 3. }\emph{En el fragmento del Quijote, ¿cuales son las 10 palabras más utilizadas? ¿Cuántas veces aparecen el artículo ``el'' y la palabra ``dijo''?}\\

Para resumir la información utilizamos codigo \texttt{Python}. Creamos un dataframe de \texttt{Pandas} con la salida facilitada por hadoop:
\begin{minted}{python}
df = pd.read_csv("salida", delimiter="\t", header=None)
\end{minted}

Para obtener las 10 palabras más frecuentes ordenamos el dataframe:
\begin{minted}{python}
df.sort_values(by=1, ascending=False, inplace=True)
print(df.head(10))
\end{minted}

Las 10 palabras mas utilizadas son: \texttt{que} (3055), \texttt{de} (2816), \texttt{y} (2585), \texttt{a} (1428), \texttt{la} (1423), \texttt{el} (1232), \texttt{en} (1155), \texttt{no} (916), \texttt{se} (753) y \texttt{los} (696).

Para buscar el número de ocurrencias de una palabra, buscamos si fila correspondiente:
\begin{minted}{python}
print(df.loc[df[0] == "el"])
print(df.loc[df[0] == "dijo"])
\end{minted}

La palabra \texttt{el} aparece un total de 1232 veces y la palabra \texttt{dijo} 272

\end{document}
